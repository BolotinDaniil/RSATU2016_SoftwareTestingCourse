\documentclass[a4paper,12pt]{article}

\usepackage[utf8x]{inputenc}
\usepackage[english, russian]{babel}

\usepackage{tabularx}
\usepackage{multirow}
\usepackage{graphicx}
\usepackage{misccorr}
\usepackage{indentfirst}


\usepackage{listings}
\usepackage{xcolor}

\usepackage{fullpage}

\usepackage[labelsep=endash,
		    margin=10pt, 
		    justification = centerlast, 
		    format = hang,
		    singlelinecheck=false
		    ]{caption}

\exhyphenpenalty=10000
\doublehyphendemerits=10000
\finalhyphendemerits=5000

\definecolor{codegreen}{rgb}{0,0.6,0}
\definecolor{codegray}{rgb}{0.5,0.5,0.5}
\definecolor{codepurple}{rgb}{0.58,0,0.82}
\definecolor{backcolour}{rgb}{0.95,0.95,0.92}
 
\lstdefinestyle{mystyle}{
    backgroundcolor=\color{backcolour},
    commentstyle=\color{codegreen},
    keywordstyle=\color{blue},
    numberstyle=\tiny\color{codegray},
    stringstyle=\color{codepurple},
    basicstyle=\footnotesize,
    breakatwhitespace=false,
    breaklines=true,
    captionpos=t,
    keepspaces=true,
    numbers=left,
    numbersep=5pt,
    showspaces=false,
    showstringspaces=false
    showtabs=false,
    tabsize=4,
    frame=tb
}
 
\lstset{style=mystyle}

\usepackage{color}
\usepackage{xcolor}
\usepackage{listings}
 
% Цвета для кода
 
\definecolor{string}{HTML}{B40000} % цвет строк в коде
\definecolor{comment}{HTML}{008000} % цвет комментариев в коде
\definecolor{keyword}{HTML}{1A00FF} % цвет ключевых слов в коде
\definecolor{morecomment}{HTML}{8000FF} % цвет include и других элементов в коде
\definecolor{сaptiontext}{HTML}{FFFFFF} % цвет текста заголовка в коде
\definecolor{сaptionbk}{HTML}{999999} % цвет фона заголовка в коде
\definecolor{bk}{HTML}{FFFFFF} % цвет фона в коде
\definecolor{frame}{HTML}{999999} % цвет рамки в коде
\definecolor{brackets}{HTML}{B40000} % цвет скобок в коде
 

%%% Отображение кода %%%
 
% Настройки отображения кода
 
\lstset{
	%morekeywords={*,...}, % если хотите добавить ключевые слова, то добавляйте	 
	% Настройки отображения     
	breaklines=true, % Перенос длинных строк
	% Для отображения русского языка
	extendedchars=true,
	literate={Ö}{{\"O}}1
	{Ä}{{\"A}}1
	{Ü}{{\"U}}1
	{ß}{{\ss}}1
	{ü}{{\"u}}1
	{ä}{{\"a}}1
	{ö}{{\"o}}1
	{~}{{\textasciitilde}}1
	{а}{{\selectfont\char224}}1
	{б}{{\selectfont\char225}}1
	{в}{{\selectfont\char226}}1
	{г}{{\selectfont\char227}}1
	{д}{{\selectfont\char228}}1
	{е}{{\selectfont\char229}}1
	{ё}{{\"e}}1
	{ж}{{\selectfont\char230}}1
	{з}{{\selectfont\char231}}1
	{и}{{\selectfont\char232}}1
	{й}{{\selectfont\char233}}1
	{к}{{\selectfont\char234}}1
	{л}{{\selectfont\char235}}1
	{м}{{\selectfont\char236}}1
	{н}{{\selectfont\char237}}1
	{о}{{\selectfont\char238}}1
	{п}{{\selectfont\char239}}1
	{р}{{\selectfont\char240}}1
	{с}{{\selectfont\char241}}1
	{т}{{\selectfont\char242}}1
	{у}{{\selectfont\char243}}1
	{ф}{{\selectfont\char244}}1
	{х}{{\selectfont\char245}}1
	{ц}{{\selectfont\char246}}1
	{ч}{{\selectfont\char247}}1
	{ш}{{\selectfont\char248}}1
	{щ}{{\selectfont\char249}}1
	{ъ}{{\selectfont\char250}}1
	{ы}{{\selectfont\char251}}1
	{ь}{{\selectfont\char252}}1
	{э}{{\selectfont\char253}}1
	{ю}{{\selectfont\char254}}1
	{я}{{\selectfont\char255}}1
	{А}{{\selectfont\char192}}1
	{Б}{{\selectfont\char193}}1
	{В}{{\selectfont\char194}}1
	{Г}{{\selectfont\char195}}1
	{Д}{{\selectfont\char196}}1
	{Е}{{\selectfont\char197}}1
	{Ё}{{\"E}}1
	{Ж}{{\selectfont\char198}}1
	{З}{{\selectfont\char199}}1
	{И}{{\selectfont\char200}}1
	{Й}{{\selectfont\char201}}1
	{К}{{\selectfont\char202}}1
	{Л}{{\selectfont\char203}}1
	{М}{{\selectfont\char204}}1
	{Н}{{\selectfont\char205}}1
	{О}{{\selectfont\char206}}1
	{П}{{\selectfont\char207}}1
	{Р}{{\selectfont\char208}}1
	{С}{{\selectfont\char209}}1
	{Т}{{\selectfont\char210}}1
	{У}{{\selectfont\char211}}1
	{Ф}{{\selectfont\char212}}1
	{Х}{{\selectfont\char213}}1
	{Ц}{{\selectfont\char214}}1
	{Ч}{{\selectfont\char215}}1
	{Ш}{{\selectfont\char216}}1
	{Щ}{{\selectfont\char217}}1
	{Ъ}{{\selectfont\char218}}1
	{Ы}{{\selectfont\char219}}1
	{Ь}{{\selectfont\char220}}1
	{Э}{{\selectfont\char221}}1
	{Ю}{{\selectfont\char222}}1
	{Я}{{\selectfont\char223}}1
	{і}{{\selectfont\char105}}1
	{ї}{{\selectfont\char168}}1
	{є}{{\selectfont\char185}}1
	{ґ}{{\selectfont\char160}}1
	{І}{{\selectfont\char73}}1
	{Ї}{{\selectfont\char136}}1
	{Є}{{\selectfont\char153}}1
	{Ґ}{{\selectfont\char128}}1
	{\{}{{{\color{brackets}\{}}}1 % Цвет скобок {
	{\}}{{{\color{brackets}\}}}}1 % Цвет скобок }
}
\begin{document}

\begin{titlepage}
\newpage


\begin{center}
	\large		
   	Министерство образования и науки Российской Федерации\\[0.5cm]
    	
	ФГБОУ ВО Рыбинский государственный авиационный технический университет имени П.А. Соловьева\\[1.0cm]

	Факультет радиоэлектроники и информатики\\[0.25cm]
		
	Кафедра математического и программного обеспечения\\ электронных вычислительных средств\\[1.5cm]
	
	\Large
	\textbf{\textsc{Курсовой проект}}\\[0.25cm]
	по  дисциплине\\
	\textbf{Тестирование и отладка\\ программного обеспечения}\\[0.5cm]
	
	по теме\\
	Тестирование программы для шифрования текстов
	
\end{center}

\vfill	
\begin{tabularx}{0.95\textwidth}{lXr}
Студенты группы ИПБ-13 			& &	Болотин Д. И.\\
								& &	Ивашин А.В. \\
Преподаватель к.т.н., ст. преп.	& & Воробьев К. А.\\
\end{tabularx}

\vspace{1.5cm}
\center Рыбинск 2016
\end{titlepage}	


\newpage
\setcounter{page}{2}

\tableofcontents

\newpage\section{Общее описание тестируемой системы}

Проект предназначен для шифрования/дешифрования текста различными методами: моноалфавитной замены, гомофонической замены, полиалфавитной замены, полиграммной замены, вертикальной перестановки с ключом,  побитовой перестановки, кодом Виженера и гоммирования с ключом. Реализоваными являются методы: моноалфавитной замены и побитовой перестановки.
Пользователь при регистрации выбирает какие методы он будет использовать во время работы с главным окном программы. Если же, выбранный им метод, не реализован, то при его выборе должно появится диалоговое окно сообщающее ему об этом.
На рис. \ref{fig:class_diagram} представлена  диаграмма классов проекта.

\par После запуска программы пользователь видит окно авторизации (рис. \ref{fig:login_form}), с помощью которого он может либо авторизаоваться: введя логин (английский алфавит) и пароль; зарегистрироваться (рис. \ref{fig:registry_form}): введя информацию о себе и выбрав предпочтительные методы шифрования - представлены на интерфейсе двумя выпадающими списками (обязательными явлюятся поля: логин, пароль и методы шифрования, если в обоих выпадающих списках выбран один и тот же метод, то в БД попадает лишь один и пользователю будет доступна работа только с одним методом); либо же завершить работу с программой с помощью кнопки "Выход".
\begin{center}
	\begin{figure}[h!]
		\centering
   		\includegraphics[scale=0.5]{img/login_form.png}
   		\caption{Окно авторизации}
   		\label{fig:login_form}
    \end{figure}
\end{center}
\begin{center}
	\begin{figure}[h!]
		\centering
   		\includegraphics[scale=0.5]{img/registry_form.png}
   		\caption{Окно регистрации}
   		\label{fig:registry_form}
    \end{figure}
\end{center}
\par После прохождения авторизации пользователю открывается главное окно приложения, в котором ему, из списка методов шифрования, будут доступны лишь те методы, которые были выбраны им на этапе регистрации (рис. \ref{fig:main_form}).
\begin{center}
	\begin{figure}[h!]
		\centering
   		\includegraphics[scale=0.5]{img/main_form.png}
   		\caption{Главное окно программы}
   		\label{fig:main_form}
    \end{figure}
\end{center}

\begin{center}
	\begin{figure}[h!]
		\centering
   		\includegraphics[scale=0.5]{img/class_diagram.png}
   		\caption{Диаграмма классов проекта}
   		\label{fig:class_diagram}
    \end{figure}
\end{center}

\subsection{Ограничения для шифруемого/расшифруемого текста}
Текст принимаемый методами шифрования состоит из строчных букв русского алфавита, из которого изключены буквы 'ё' и 'й', а так же пробел. Оставшиеся символы не входят в состав открытого алфавита, поэтому при их использовании в тексте пользователь увидит следующее сообщение: "Недопустимый символ. Расшифровка/Шифрование не возможно" (Последнее предложение зависит от выбранного пользователем метода. Пример на рис. \ref{fig:wrong_char_and_response}.
\begin{center}
	\begin{figure}[h!]
		\centering
		\includegraphics[scale=0.7]{img/wrong_char.png}
		\includegraphics[scale=0.7]{img/wrong_char_response.png}
		\caption{Не верный символ в тексте}
		\label{fig:wrong_char_and_response}
	\end{figure}
\end{center}

\newpage\section{Общее описание тестирования}
В ходе данного тестирования проверяется работа пользовательского GUI как автоматизирванно (с помощью TestFx) так и в ручную. Проверяется работа приложения с базой данных, а также сохранение/загрузка текста в файл.

\par Для проведения тестирования использована библиотека JUnit, а также TestFX, которая нужна для моделирования взаимодействия пользователя с интерфейсом. Для проверки содержимого БД использовался просмотрщик БД ИСР IntellijIdea.

\newpage \section{Модульное тестирование}
\subsection{Общее описание}
Целью модульного тестирования является проверка корректности работы отдельных функций классов программы.
Для его проведения использована библиотека jUnit, а также TestFX для моделирования работы полльзователя с интерфейсом программы (GUI). При интеграционном тестирование были использованы те же библиотеки, но при этом моделировалось взаимодействие классов программы. Поэтому для проведения этого этапа тестирования были разработаны классы для тестирования GUI с обобщеным наборов методов, набор данных классов представлен на риунке \ref{fig:class_diagram_tests_gui}. Атрибуты и методы классов упущены на данной диагремме.

\begin{center}
	\begin{figure}[h!]
		\centering
		\includegraphics[scale=0.6]{img/class_diagram_ui_tests.png}
		\caption{Диаграмма классов для тестирования GUI}
		\label{fig:class_diagram_tests_gui}
	\end{figure}
\end{center}

Для модульного тестирования классов шифрования, баззы данных, дополнительных функций для работы программы были разработы дополнительные классы: BitReversCipherTest, MonoAlphabetCipherTest, DataBaseTesting и UtilFunctionsTest.

\newpage\subsection{Описание методов и тестов класса MonoAlphabetCipher}
\subsubsection{Метод createOpenAlphabet}
Метод инициализирует массив допустимых символов. Сначала должен быть пробел, потом символы 'a' - 'я' заисключением символов 'ё' и 'й'. Исходный код представлен в листинге \ref{listing:createOpenAlphabet}. Метод унаследован их базового класса.
\begin{lstlisting}[language=java, caption=метод createOpenAlphabet, label = listing:createOpenAlphabet]
	private void createOpenAlphabet() {
        openAlphabet[0] = '\u0020';
        for (int i = 0; i < 9; i++) {
            openAlphabet[i + 1] = (char) ('а' + i);
        }
        for (int i = 10; i < 32; ++i) {
            openAlphabet[i] = (char) ('а' + i);
        }
    }
\end{lstlisting}
Необходимо проверить привильность инициализации, это сделано в листинге \ref{listing:testCalculationPrivateAlphabet}
\subsection{Метод calculationPrivateAlphabet}
Вычисляет закрытый алфавит на основании ключа шифрования (целое число, не превышающее мо модулю $10^9$) (листинг \ref{listing:calculationPrivateAlphabet}). Затем при шифровании символ, имеющий позицию i в OpenAlphabet заменяется символом с этим же номером из массива PrivetAlphabet.
\begin{lstlisting}[language=java, caption=метод calculationPrivateAlphabet, label = listing:calculationPrivateAlphabet]
	@Override
    public void calculationPrivateAlphabet(Integer key) {
        for (int i = 0; i < 32; i++) {
            privateAlphabet[i] = openAlphabet[Math.floorMod(i + key, 32)];
        }
    }
    
\end{lstlisting}
Необходимо проверитть правильность генерации закрытого и открытого алфавита. это делает следующий метод (листинг \ref{listing:testCalculationPrivateAlphabet}). Закрый алфавит созданый по ключу 0 должен совпадать с открытым ключом.
\begin{lstlisting}[language=java, caption=метод теста testCalculationPrivateAlphabet, label = listing:testCalculationPrivateAlphabet]
@Test
    public void testCalculationPrivateAlphabet() {
        char[] actuals = {' ', 'а', 'б', 'в', 'г', 'д', 'е', 'ж', 'з', 'и', 'к', 'л', 'м', 'н', 'о', 'п', 'р', 'с', 'т', 'у', 'ф', 'х', 'ц', 'ч', 'ш', 'щ', 'ъ', 'ы', 'ь', 'э', 'ю', 'я'};
        monoAlphabetCipher.calculationPrivateAlphabet(0);
        assertArrayEquals(monoAlphabetCipher.getPrivateAlphabet(), actuals);
        assertArrayEquals(monoAlphabetCipher.getOpenAlphabet(), actuals);
        assertArrayEquals(monoAlphabetCipher.getOpenAlphabet(), monoAlphabetCipher.getOpenAlphabet());
    }
\end{lstlisting}
Далее проверка генерации закрытого алфавита при ненулевом ключе (листинг \ref{listing:testCalculationPrivateAlphabetCaesarCipher}).
\begin{lstlisting}[language=java, caption=метод теста testCalculationPrivateAlphabet, label = listing:testCalculationPrivateAlphabetCaesarCipher]
	@Test
    public void testCalculationPrivateAlphabetCaesarCipher() {
        char[] actuals = {'в', 'г', 'д', 'е', 'ж', 'з', 'и', 'к', 'л', 'м', 'н', 'о', 'п', 'р', 'с', 'т', 'у', 'ф', 'х', 'ц', 'ч', 'ш', 'щ', 'ъ', 'ы', 'ь', 'э', 'ю', 'я', ' ', 'а', 'б'};
        monoAlphabetCipher.calculationPrivateAlphabet(3);
        assertArrayEquals(monoAlphabetCipher.getPrivateAlphabet(), actuals);
\end{lstlisting}
Тестируем возвращает ли метод исключение при некорректном значении ключа (листинг \ref{listing1:testExceptions}).
\begin{lstlisting}[language=java, caption=методы проверки исключений, label = listing1:testExceptions]
	@Test(expected = NullPointerException.class)
    public void testNullCalculationPrivateAlphabet() {
        monoAlphabetCipher.calculationPrivateAlphabet(null);
    }

    @Test(expected = ClassCastException.class)
    public void testWrongArgumentCalculationPrivateAlphabet() {
        monoAlphabetCipher.calculationPrivateAlphabet("test");
    }
\end{lstlisting}
\subsubsection{Метод encodeText и decodeText}
Выполняют шифрование и дешифрование текста. Эти методы унаследованы из базового класса (листинг \ref{listing1:encodeAndDecode}). Если переданная строка содержит символы, недопустимые в данном алфавите, то метод должен вернуть строку "Недопустимый символ. Шифрование/Дешифрование не возможно".
\begin{lstlisting}[language=java, caption=методы encode и decode, label = listing1:encodeAndDecode]
     /**
     * Закодировать текст
     *
     * @param originalText текст
     * @return закодированный текст
     */
    public String encodeText(String originalText) {
        originalText = originalText.replaceAll("\n", " ").toLowerCase();
        char[] text = originalText.toCharArray();
        char[] result = new char[text.length];
        int index;
        for (int i = 0; i < text.length; i++) {
            index = contains(openAlphabet, text[i]);
            if (index < 0) return "Недопустимый символ. Шифрование не возможно";
            result[i] = privateAlphabet[index];
        }
        return String.valueOf(result);
    }

    /**
     * Раскодировать текст
     *
     * @param originalText текст
     * @return раскодированный текст
     */
    public String decodeText(String originalText) {
        char[] text = originalText.toCharArray();
        char[] result = new char[text.length];
        int index;
        for (int i = 0; i < text.length; i++) {
            index = contains(privateAlphabet, text[i]);
            if (index < 0) return "Недопустимый символ. Расшифровка не возможно";
            result[i] = openAlphabet[index];
        }
        return String.valueOf(result);
    }

    private int contains(char[] chars, char symbol) {
        for (int i = 0; i < chars.length; i++) {
            if (chars[i] == symbol) {
                return i;
            }
        }
        return -1;
    }
\end{lstlisting}
Необходимо проверить корректность шифрования. Это делают следующие методы (листинг \ref{listing1:testEncocodeAndDecode}).
\begin{lstlisting}[language=java, caption=методы проверки корректности шифрования/дешифрования, label = listing1:testEncocodeAndDecode]
    @Test
    public void testEncodeText() {
        monoAlphabetCipher.calculationPrivateAlphabet(3);
        assertEquals(monoAlphabetCipher.encodeText("тест"), "хифх");
    }

    @Test
    public void testEncodeAndDecode() {
        monoAlphabetCipher.calculationPrivateAlphabet(30);
        String encode = monoAlphabetCipher.encodeText("тест");
        assertEquals("тест", monoAlphabetCipher.decodeText(encode));
    }

\end{lstlisting}
А также корректность певедения метода при недопустимой строке (листинг \ref{listing1:testWrongCharInEncodeAndDecodeText}).
\begin{lstlisting}[language=java, caption=методы проверки корректности шифрования/дешифрования, label = listing1:testWrongCharInEncodeAndDecodeText]
    @Test
    public void testWrongCharInEncodeAndDecodeText() {
        monoAlphabetCipher.calculationPrivateAlphabet(1);
        Assert.assertEquals(monoAlphabetCipher.encodeText("tttt"), "Недопустимый символ. Шифрование не возможно");
        Assert.assertEquals(monoAlphabetCipher.encodeText("й"), "Недопустимый символ. Шифрование не возможно");
        Assert.assertEquals(monoAlphabetCipher.encodeText("ё"), "Недопустимый символ. Шифрование не возможно");
        Assert.assertEquals(monoAlphabetCipher.encodeText("-"), "Недопустимый символ. Шифрование не возможно");
        Assert.assertEquals(monoAlphabetCipher.encodeText("+"), "Недопустимый символ. Шифрование не возможно");
        Assert.assertEquals(monoAlphabetCipher.decodeText("tttt"), "Недопустимый символ. Расшифровка не возможно");
        Assert.assertEquals(monoAlphabetCipher.decodeText("й"), "Недопустимый символ. Расшифровка не возможно");
        Assert.assertEquals(monoAlphabetCipher.decodeText("ё"), "Недопустимый символ. Расшифровка не возможно");
        Assert.assertEquals(monoAlphabetCipher.decodeText("-"), "Недопустимый символ. Расшифровка не возможно");
        Assert.assertEquals(monoAlphabetCipher.decodeText("+"), "Недопустимый символ. Расшифровка не возможно");
    }
\end{lstlisting}
\newpage\subsection{Описание методов и тестов класса BitReverseCipher}
Данный класс выполняет шифрование/дешифрование методом побитовой перестановки. Ключём шифрования является строка являющаяся перестановкий "12345", которая задаёт правило перестановки битов для каждого символа. Напимер строка "54312" означает, что 5-й бит будет переставлен на 1 место (нумерация от 1 до 5), 4-й бит на 2-e место, 3-й бит остаётся на месте, 1-й бит идёт на 4 место, 2-й бит на 5-е.
\par Далее описаны методы, которые переопределены у базового класса.
\subsubsection{Метод findPosition}
Метод на основе ключа генерирует массив, с правилами переставки битов в исходных символов (листинг \ref{listing2:findPosition}).
\begin{lstlisting}[language=java, caption=метод findPosition, label=listing2:findPosition]
    private int[] findPosition(String arrayKey) {
        int[] result = new int[arrayKey.length()];
        result[0] = arrayKey.indexOf("1");
        result[1] = arrayKey.indexOf("2");
        result[2] = arrayKey.indexOf("3");
        result[3] = arrayKey.indexOf("4");
        result[4] = arrayKey.indexOf("5");
        return result;
    }
\end{lstlisting}
Корректнгость данного метода проверяется при описанного дальше метода.
\subsubsection{Метод calculationPrivateAlphabet}
Метод генерирует закрытый алфавит. На вход подаётся строка с ключём (листинг \ref{listing2:calculationPrivateAlphabet}).
\begin{lstlisting}[language=java, caption=метод calculationPrivateAlphabet, label=listing2:calculationPrivateAlphabet]
    public void calculationPrivateAlphabet (String key) {
        int[] arrayKey = findPosition(key);
        for (int i = 0; i < 32; i++) {
            StringBuilder stI = new StringBuilder(Integer.toBinaryString(i));
            for (int j = stI.length(); j < 5; j++) {
                stI.insert(0, "0");
            }
            StringBuilder charAlph = new StringBuilder();
            char[] mass = stI.toString().toCharArray();
            for (int anArrayKey : arrayKey) {
                charAlph.append(mass[anArrayKey]);
            }
            int exitI = Integer.parseInt(charAlph.toString(), 2);
            privateAlphabet[i] = openAlphabet[exitI];
        }
    }
\end{lstlisting}

Можно заметить, что шифрование пробела всегда приведёт к пробельному символу(код 0). Для проверки этого разработаен следующий метод (листинг \ref{listing2:testFirstElementPrivateAlphabetAtAnyKey}).
\begin{lstlisting}[language=java, caption=метод calculationPrivateAlphabet, label=listing2:testFirstElementPrivateAlphabetAtAnyKey]
    @Test
    public void testFirstElementPrivateAlphabetAtAnyKey() {
        bitReversCipher.calculationPrivateAlphabet("54321");
        assertEquals(bitReversCipher.getPrivateAlphabet()[0], ' ');
        bitReversCipher.calculationPrivateAlphabet("53241");
        assertEquals(bitReversCipher.getPrivateAlphabet()[0], ' ');
        bitReversCipher.calculationPrivateAlphabet("13425");
        assertEquals(bitReversCipher.getPrivateAlphabet()[0], ' ');
        bitReversCipher.calculationPrivateAlphabet("31245");
        assertEquals(bitReversCipher.getPrivateAlphabet()[0], ' ');
    }
\end{lstlisting}

Далее проверить в целом построение закрытого и открытого алфавита (листинг \ref{listing2:testCalculationPrivateAlphabet}).
\begin{lstlisting}[language=java, caption=метод calculationPrivateAlphabet, label=listing2:testCalculationPrivateAlphabet]
    @Test
    public void testCalculationPrivateAlphabetEqualsOpenAlphabet() {
        char[] actuals = {' ', 'а', 'б', 'в', 'г', 'д', 'е', 'ж', 'з', 'и', 'к', 'л', 'м', 'н', 'о', 'п', 'р', 'с', 'т', 'у', 'ф', 'х', 'ц', 'ч', 'ш', 'щ', 'ъ', 'ы', 'ь', 'э', 'ю', 'я'};
        bitReversCipher.calculationPrivateAlphabet("12345");
        assertArrayEquals(bitReversCipher.getPrivateAlphabet(), actuals);
        assertArrayEquals(bitReversCipher.getOpenAlphabet(), actuals);
        assertArrayEquals(bitReversCipher.getOpenAlphabet(), bitReversCipher.getOpenAlphabet());
    }

    @Test
    public void testCalculationPrivateAlphabet() {
        char[] actuals = {' ', 'р', 'з', 'ш', 'г', 'ф', 'м', 'ь', 'б', 'т', 'к', 'ъ', 'е', 'ц', 'о', 'ю', 'а', 'с', 'и', 'щ', 'д', 'х', 'н', 'э', 'в', 'у', 'л', 'ы', 'ж', 'ч', 'п', 'я'};
        bitReversCipher.calculationPrivateAlphabet("54321");
        assertArrayEquals(bitReversCipher.getPrivateAlphabet(), actuals);
    }
\end{lstlisting}
Затем можно проверить, что при некорректном значии ключа происходят исключения (листинг \ref{listing2:testWrongValueForCalculationPrivateAlphabet}).
\begin{lstlisting}[language=java, caption=метод calculationPrivateAlphabet, label=listing2:testWrongValueForCalculationPrivateAlphabet]
    @Test(expected = IndexOutOfBoundsException.class)
    public void testWrongValueForCalculationPrivateAlphabet() {
        bitReversCipher.calculationPrivateAlphabet("00000");
    }

    @Test(expected = ClassCastException.class)
    public void testWrongArgumentForCalculationPrivateAlphabet() {
        bitReversCipher.calculationPrivateAlphabet(1213234);
    }
\end{lstlisting}

\subsubsection{Тестирование остальных методов}

Далее необходимо проверить правальность шифрования и расшифрования строки (листинг \ref{listing2:testEncodeDecode}).
\begin{lstlisting}[language=java, caption=метод calculationPrivateAlphabet, label=listing2:testEncodeDecode]
	@Test
    public void testEncode() {
        bitReversCipher.calculationPrivateAlphabet("54321");
        assertEquals(bitReversCipher.encodeText("тест"), "имси");
    }

    @Test
    public void testEncodeDecode() {
        bitReversCipher.calculationPrivateAlphabet("54321");
        String encode = bitReversCipher.encodeText("тест");
        assertEquals(bitReversCipher.decodeText(encode), "тест");
    }
\end{lstlisting}
Проверка поведения при некоректном шифруемом/дешифруемом тексте (листинг \ref{listing2:testWrongCharInEncodeAndDecodeText}).
\begin{lstlisting}[language=java, caption=метод calculationPrivateAlphabet, label=listing2:testWrongCharInEncodeAndDecodeText]
    @Test
    public void testWrongCharInEncodeAndDecodeText() {
        bitReversCipher.calculationPrivateAlphabet("54321");
        Assert.assertEquals(bitReversCipher.encodeText("tttt"), "Недопустимый символ. Шифрование не возможно");
        Assert.assertEquals(bitReversCipher.encodeText("й"), "Недопустимый символ. Шифрование не возможно");
        Assert.assertEquals(bitReversCipher.encodeText("ё"), "Недопустимый символ. Шифрование не возможно");
        Assert.assertEquals(bitReversCipher.encodeText("-"), "Недопустимый символ. Шифрование не возможно");
        Assert.assertEquals(bitReversCipher.encodeText("+"), "Недопустимый символ. Шифрование не возможно");
        Assert.assertEquals(bitReversCipher.decodeText("tttt"), "Недопустимый символ. Расшифровка не возможно");
        Assert.assertEquals(bitReversCipher.decodeText("й"), "Недопустимый символ. Расшифровка не возможно");
        Assert.assertEquals(bitReversCipher.decodeText("ё"), "Недопустимый символ. Расшифровка не возможно");
        Assert.assertEquals(bitReversCipher.decodeText("-"), "Недопустимый символ. Расшифровка не возможно");
        Assert.assertEquals(bitReversCipher.decodeText("+"), "Недопустимый символ. Расшифровка не возможно");
    }
\end{lstlisting}

\subsection{Описание методов и тестов класса UtilFunctions}

Данный класс реализует функции, необходимые во многих модулях проекта. А именно проверку строки на пустоту и MD5 хеширование(применяется для хранения паролей). Необходимо проверить эту функциональнсть, для чего и разработан класс UtilFunctionsTest.
\par Тестирование md5 представлено в листинге \ref{listing:testMd5}.

\begin{lstlisting}[language=java, caption=тестирование md5 , label=listing:testMd5]
    @Test
    public void md5EqualsTesting() {
        String expected = md5Custom("1");
        assertEquals(expected, "c4ca4238a0b923820dcc509a6f75849b");
        expected = md5Custom("md5test");
        assertEquals(expected, "82da61aa724b5d149a9c5dc8682c2a45");
        expected = md5Custom("абра-кадбра, магия какая-то");
        assertEquals(expected, "783e47ca544ddf34ccce81b524adbbe3");
    }

    @Test
    public void md5NullArgument() {
        assertEquals(md5Custom(null), "");
    }

\end{lstlisting}


Тестирование метода проверки строки на пустоту представлено в листинге \ref{listing:testIsNullString}

\begin{lstlisting}[language=java, caption=тестирование isNullString, label=listing:testIsNullString]
    @Test
    public void isNullStringTest() {
        assertEquals(isNullString(null), true);
        assertEquals(isNullString(""), true);
        assertEquals(isNullString("          "), true);
        assertEquals(isNullString("    dsfsdf   "), false);
        assertEquals(isNullString("тестовое значение"), false);
        assertEquals(isNullString(" "), true);
    }
\end{lstlisting}

\subsection{Описание тестиования класса RegistryController}
Данный класс отвечает за окно регистрации пользователя и за взаимодействие с БД во время регистрации.
Тестирование осуществлялось с помощью моделирования рботы пользователя с GUI (TestFX). Для регистрации необходимо как минимум заполнить поля "Логин" и "Пароль", поэтому необходимо протестировать, что при попытке регистрации не указав хотя бы одно из этих полей будет выведено соответствующее сообщение. Это тестирование выполняет класс RegistryUiTest (листинг \ref{listing:testNullFields}). Также необходимо протестировать поведение программы при попытке зарегистрировать пользователя, логин которого совпадает с зарегестрированным ранее. Это выполняет класс DatabaseAndRegistryUiTest (листинг \ref{listing:testRegExistUser}).

\begin{lstlisting}[language=java, caption=тестирование не заполнения полей при регистрации, label=listing:testNullFields]
public class RegistryUiTest extends GuiTest {
    private static FxRobot robot;
    private Label outputErrorText;

    @BeforeClass
    public static void createRobot() {
        robot = new FxRobot();
    }

    @Before
    public void getOutputErrorText() {
        outputErrorText =  find("#outputErrorText");
    }

    @After
    public void clickResetField() {
        robot.clickOn("#reset");
    }

    @Override
    protected Parent getRootNode() {
        Parent parent = null;
        try {
            parent = FXMLLoader.load(getClass().getResource("/fxml/Registry.fxml"));
            return parent;
        } catch (IOException ex) {
            // TODO ...
        }
        return parent;
    }


    @Test
    public void emptyLoginAndPasswordFieldTest() {
        robot.clickOn("#signUp").sleep(500);
        Assert.assertEquals(outputErrorText.getText(), "Не заполнены обязательные поля: Логин; Пароль; ");
    }

    @Test
    public void emptyLoginFieldTest() {
        robot.clickOn("#login").write("notEmptyLogin");
        robot.clickOn("#signUp");
        Assert.assertEquals(outputErrorText.getText(), "Не заполнены обязательные поля: Пароль; ");
    }

    @Test
    public void emptyPasswordFieldTest() {
        robot.clickOn("#password").write("tesPass");
        robot.clickOn("#signUp");
        Assert.assertEquals(outputErrorText.getText(), "Не заполнены обязательные поля: Логин; ");
    }
}

\end{lstlisting}

\begin{lstlisting}[language=java, caption=тестирование попытки регистрации существующего пользователя, label=listing:testRegExistUser]
public class DatabaseAndRegistryUiTest extends LoginUiCustomTest {

    @Test
    public void userExistsTest() {
        clickOn(registry);
        clickOn("#login").write("Aleksey");
        clickOn("#password").write("tesPass");
        clickOn("#signUp");
        Label outputErrorText = find("#outputErrorText");
        Assert.assertEquals(outputErrorText.getText(), "Пользователь с логином: Aleksey существует!");
        closeRegistryWindowAndTest();
    }


}

\end{lstlisting}

\subsection{Описание тестирования класса MainWindowController}
Данный класс отвечает за главное окно программы. Необходимо проверить, корректность его работы. Это сделано в классе MainWindowUiTest, который описан в листинге \ref{listing:testMainWindow}. А именно проверить корректность работы поля memo, работу кнопки "Очистить", и что по умолчанию не доступен ни один метод шифрования и дешифрования.

\begin{lstlisting}[language=java, caption=тестирование MainWindow, label=listing:testMainWindow]
public class MainWindowUiTest extends CustomMainWindowUiTest {
    private TextArea textArea;

    @Before
    public void findTextArea() {
        textArea = find("#textArea");
    }

    @Override
    protected Parent getRootNode() {
        Parent parent = null;
        try {
            parent = FXMLLoader.load(getClass().getResource("/fxml/MainWindow.fxml"));
            return parent;
        } catch (IOException ex) {
            // TODO ...
        }
        return parent;
    }

    @Test
    public void writeInTextAreaTest() {
        robot.clickOn(textArea).write("авыаыва");
        verifyThat(textArea, hasText("авыаыва"));
    }

    @Test
    public void clickAllEncodeMenuTest() {
        robot.clickOn("#encodeMenu");
        clickAllMethodsMenu("encode");
        robot.clickOn("#encodeMenu");
    }

    @Test
    public void clickAllDecodeMenuTest() {
        robot.clickOn("#decodeMenu");
        clickAllMethodsMenu("decode");
        robot.clickOn("#decodeMenu");
    }

    private static void clickAllMethodsMenu(String variantName) {
        String itemId;
        for (String name : methodNames) {
            itemId = "#" + variantName + name;
            robot.clickOn(itemId);
        }
    }
    @Test
    public void clearTextArea() {
        String testText = "текстовый текст, для проверки очистки области ввода текста";
        robot.clickOn(textArea).write(testText);
        verifyThat(textArea, hasText(testText));
        robot.clickOn("#fileMenu").clickOn("#clear");
        verifyThat(textArea, hasText(""));
    }
}
\end{lstlisting}

\subsection{Результаты тестирования и выводы на данном этапе}
Результаты прохождения тестов представлены в таблице 

Тут можно запилить такую табличку. Как считаешь?

\begin{table}[h]
	\caption{Тесты С1}
	\centering
	\begin{tabular}{|c|c|c|}
	\hline 
	№ & Тестирующий класс: название теста & Результат \\ 
	\hline 
	1 & MainWindowUiTest: writeInTextAreaTest & Успешно пройден  \\
	\hline 
\end{tabular} 
\label{table:tests_C1} 
\end{table}


\newpage \section{Интеграционное тестирование}
\subsection{Общее описание}
Целью интеграционного тестирования является проверка всех возможных непосредвенных взаимодействий классов программы друг с другом. 
Для его проведения использованы следующие библиотеки: JUnit и TestFX.

\newpage\subsection{Тестирование взаимодействия класса Login}
Login - это главный класс приложения, который запускает его и одновременно является контроллером для окна авторизации. Пользователь должен пройти авторизацию или зарегистрироваться. Если после третьей попытки не удалось авторизиваться, то программа закрывается.
\subsubsection{Закрытие программы при неверной авторизации}
С помощью TestFx моделируется попытка неудачной авторизации: три раза вводим не верный пароль. На каждом шаге проверяем появилось ли в специальной строке сообщение об ошибке: "Не верный пароль''", выделенное красным цветом. (листинг \ref{listing_login:test_wrong_authorization}).
\begin{lstlisting}[language=java, caption=Тестирование неверной авторизации, label=listing_login:test_wrong_authorization]
public class LoginUiCloseTest extends LoginUiCustomTest {
    @Test
    public void threeWrongAuthorizationTest() {
        for (int i = 0; i < 3; i++) {
            clickOn(userName).write("Aleksey");
            clickOn(password).write("tttt");
            clickOn(authorization).sleep(100);
            verifyThat(resultAuthorization, hasText("Не верный пароль"));
            assertEquals(resultAuthorization.getFill(), Color.FIREBRICK);
            userName.clear();
            password.clear();
        }
        assertEquals(stage.isShowing(), false);
        assertEquals(sessionFactory.isClosed(), true);
    }
}
\end{lstlisting}

\subsubsection{Реакция программы при неверном пароле}
При вводе корректного (существующего) логина пользователя, но некорректного пароля для него в специальной строке формы авторизации должна появиться надпись "Неверный пароль''", выделенная красным цветом. Этот (листинг \ref{listing_login:test_wrong_password}) и последующие тесты класса Login описаны в классе LoginUiTest.

\begin{lstlisting}[language=java, caption=Тестирование ввода неверного пароля, label=listing_login:test_wrong_password]      
    @Test
    public void wrongAuthorizationTest() {
        assertEquals(sessionFactory.isClosed(), false);
        clickOn(userName).write("Aleksey");
        clickOn(password).write("retdfgdfh");
        clickOn(authorization);
        verifyThat(resultAuthorization, hasText("Не верный пароль"));
        assertEquals(resultAuthorization.getFill(), Color.FIREBRICK);
    }
\end{lstlisting}

\subsubsection{Реакция программы при не существующем логине}
После нажатия на кнопку авторизации в БД ищется пользователь с указаным логином, если не возвращается пользователь с таким логином, то должно появиться сообщение: "Такого пользователя не существует" (листинг \ref{listing_login:test_wrong_user}).
\begin{lstlisting}[language=java, caption=Тестирование при несуществующем логине, label=listing_login:test_wrong_user]
    @Test
    public void thisUserDoesntExistTest() {
        assertEquals(sessionFactory.isClosed(), false);
        clickOn(userName).write("ТестНик");
        clickOn(password).write("tttt");
        clickOn(authorization);
        verifyThat(resultAuthorization, hasText("Такого пользователя не существует"));
        assertEquals(resultAuthorization.getFill(), Color.FIREBRICK);
    }
\end{lstlisting}


\subsubsection{Реакция программы при верной авторизации}
При прохождении авторизации окно авторизации должно закрыться, а вместо него должно появиться основное окно программы (листинг \ref{listing_login:test_true_autorization}). 

\begin{lstlisting}[language=java, caption=Тестирование прохождения авторизации, label=listing_login:test_true_autorization]
    @Test
    public void openMainWindowTest() {
        assertEquals(sessionFactory.isClosed(), false);
        clickOn(userName).write("Aleksey");
        clickOn(password).write("yui");
        clickOn(authorization);
        verifyThat(resultAuthorization, hasText("Пароль верный"));
        assertEquals(resultAuthorization.getFill(), Color.GREEN);
        waitUntil("#AnchorPane", visible());
    }
\end{lstlisting}

\subsubsection{Тестирование перехода к окну регистрации}
При нажатии на кнопку "'Регистация'' , должно появится соответствующее окно, а окно для авторизации исчезнуть (листинг \ref{listing_login:test_registration_to_registration}). 
\begin{lstlisting}[language=java, caption=Тестирование перехода к окну регистрации, label=listing_login:test_registration_to_registration]
	protected void openRegistryWindowAndTest() {
        clickOn(registry);
        assertEquals(gridPane.isDisable(), true);
        waitUntil("#registryPane", visible());
    }

    protected void closeRegistryWindowAndTest() {
        clickOn("#back");
        assertEquals(gridPane.isDisable(), false);
    }     
    @Test
    public void openRegistryWindowTest() {
        openRegistryWindowAndTest();
        closeRegistryWindowAndTest();
    }
\end{lstlisting}

\newpage\subsubsection{Тестирование взаимодействия классов \\MonoAlphabetCipher и BitReverseCipher}
Тестирование осуществляется посредством многократного шифрования и расшифрования текста различными методами. После последовательного шифрования и расшифрования разными методами с неизменными ключами текст должен совпасть с исходным. Если провести такой же тест, но при расшифровании изменить ключ одного из методов, то результат должен не совпасть с исходным. Далее проведём тест, при котором зашуфруем несколько раз обоими методами с разнами ключами и расшуфруем в обратном порядке с соответствующими ключами, результат должен совпасть с исходным. Все эти три теста описаны в классе IntegrationCipherTest, который представлен в листинге \ref{listing_cipher:test_cipher_methods}.

\begin{lstlisting}[language=java, caption=класс IntegrationCipherTest, label=listing_cipher:test_cipher_methods]
public class IntegrationCipherTest {
    private static SubstitutionCipher<Integer> monoAlphabet;
    private static SubstitutionCipher<String> bitRevers;
    private String actualText;

    @BeforeClass
    public static void test() {
        monoAlphabet = new MonoAlphabetCipher();
        bitRevers = new BitReversCipher();
    }

    @Before
    public void createCipherClass() {
        actualText = "специальныи текст для тестов";
    }

    @Test
    public void integrationCipherEqualsTest() {
        monoAlphabet.calculationPrivateAlphabet(4);
        bitRevers.calculationPrivateAlphabet("32451");
        String expectedText = monoAlphabet.encodeText(actualText);
        expectedText = bitRevers.decodeText(expectedText);
        expectedText = bitRevers.encodeText(expectedText);
        expectedText = monoAlphabet.decodeText(expectedText);
        Assert.assertEquals(expectedText, actualText);
    }

    @Test
    public void  integrationCipherNotEqualsTest() {
        monoAlphabet.calculationPrivateAlphabet(4);
        bitRevers.calculationPrivateAlphabet("32451");
        String testText = monoAlphabet.encodeText(actualText);
        testText = bitRevers.decodeText(testText);
        bitRevers.calculationPrivateAlphabet("52341");
        String expectedText = bitRevers.encodeText(testText);
        Assert.assertNotEquals(expectedText, testText);
    }

    @Test
    public void  integrationDeepCipherEqualsTest() {
        monoAlphabet.calculationPrivateAlphabet(2);
        String expectedText = monoAlphabet.encodeText(actualText);

        bitRevers.calculationPrivateAlphabet("53241");
        expectedText = bitRevers.encodeText(expectedText);
        Assert.assertNotEquals(expectedText, actualText);

        monoAlphabet.calculationPrivateAlphabet(3);
        expectedText = monoAlphabet.decodeText(expectedText);

        bitRevers.calculationPrivateAlphabet("24531");
        expectedText = bitRevers.decodeText(expectedText);
        Assert.assertNotEquals(expectedText, actualText);

        expectedText = bitRevers.encodeText(expectedText);
        expectedText = monoAlphabet.encodeText(expectedText);
        Assert.assertNotEquals(expectedText, actualText);

        bitRevers.calculationPrivateAlphabet("53241");
        expectedText = bitRevers.decodeText(expectedText);

        monoAlphabet.calculationPrivateAlphabet(2);
        expectedText = monoAlphabet.decodeText(expectedText);

        Assert.assertEquals(expectedText, actualText);
    }
}
\end{lstlisting}


\newpage\subsubsection{Тестирование взаимодействия класса MainWindow}
MainWindow - класс отвечающий за главное окно программы. Он взаимодействует с классами, реализующими методы шифрования. Проведём тестирование, аналогичное тестированию классов MonoAlphabetCipher и BitReverseCipher, но теперь классы для шифрования будут взаимодействовать не напрямую, а через MainWindow (как и происходит на самом деле), посредством моделирвания работы пользователя с gui. Оно представлено в листинге \ref{listing_mainWindow:MainWindowUiIntegrationTestn}. Предыдушее тестирование направлено на проверку корректности классов шифрования и их взаимодействия, а такущее преимущественно на корректность работы MainWindow с данными классами.

\begin{lstlisting}[language=java, caption=класс MainWindowUiIntegrationTest, label=listing_mainWindow:MainWindowUiIntegrationTestn]
public class MainWindowUiIntegrationTest extends CustomMainWindowUiTest {
    private TextArea textArea;
    String actualText;

    @BeforeClass
    public static void mainWindowSettings() {
        MainWindowController.nameMethods = Arrays.asList(
                methodNames.get(0), methodNames.get(5)
        );
    }

    @Before
    public void findTextArea() {
        textArea = find("#textArea");
        actualText = "специальныи текст для тестов";
        robot.clickOn(textArea).write(actualText);
    }

    @After
    public void clearTextArea() {
        textArea.clear();
    }

    @Test
    public void integrationCipherEqualsTest() {
        robot.clickOn("#encodeMenu").clickOn("#encodeMonoAlphabet")
                .clickOn("#txtFieldDialog").write('4').clickOn("#okDialog");
        robot.clickOn("#decodeMenu").clickOn("#decodeBitRevers")
                .clickOn("#txtFieldDialog").write("32451").clickOn("#okDialog");
        robot.clickOn("#encodeMenu").clickOn("#encodeBitRevers")
                .clickOn("#txtFieldDialog").write("32451").clickOn("#okDialog");
        robot.clickOn("#decodeMenu").clickOn("#decodeMonoAlphabet")
                .clickOn("#txtFieldDialog").write('4').clickOn("#okDialog");
        Assert.assertEquals(textArea.getText(), actualText);
    }

    @Test
    public void integrationCipherNotEqualsTest() {
        robot.clickOn("#encodeMenu").clickOn("#encodeMonoAlphabet")
                .clickOn("#txtFieldDialog").write('4').clickOn("#okDialog");
        robot.clickOn("#decodeMenu").clickOn("#decodeBitRevers")
                .clickOn("#txtFieldDialog").write("32451").clickOn("#okDialog");
        String testText = textArea.getText();
        robot.clickOn("#decodeMenu").clickOn("#decodeBitRevers")
                .clickOn("#txtFieldDialog").write("52341").clickOn("#okDialog");
        Assert.assertNotEquals(textArea.getText(), testText);
    }

    @Test
    public void  integrationDeepCipherEqualsTest() {
        robot.clickOn("#encodeMenu").clickOn("#encodeMonoAlphabet")
                .clickOn("#txtFieldDialog").write("2").clickOn("#okDialog");
        robot.clickOn("#encodeMenu").clickOn("#encodeBitRevers")
                .clickOn("#txtFieldDialog").write("53241").clickOn("#okDialog");
        Assert.assertNotEquals(textArea.getText(), actualText);
        robot.clickOn("#decodeMenu").clickOn("#decodeMonoAlphabet")
                .clickOn("#txtFieldDialog").write('3').clickOn("#okDialog");
        robot.clickOn("#decodeMenu").clickOn("#decodeBitRevers")
                .clickOn("#txtFieldDialog").write("24531").clickOn("#okDialog");
        Assert.assertNotEquals(textArea.getText(), actualText);
        robot.clickOn("#encodeMenu").clickOn("#encodeBitRevers")
                .clickOn("#txtFieldDialog").write("24531").clickOn("#okDialog");
        robot.clickOn("#encodeMenu").clickOn("#encodeMonoAlphabet")
                .clickOn("#txtFieldDialog").write('3').clickOn("#okDialog");
        Assert.assertNotEquals(textArea.getText(), actualText);
        robot.clickOn("#decodeMenu").clickOn("#decodeBitRevers")
                .clickOn("#txtFieldDialog").write("53241").clickOn("#okDialog");
        robot.clickOn("#decodeMenu").clickOn("#decodeMonoAlphabet")
                .clickOn("#txtFieldDialog").write('2').clickOn("#okDialog");
        Assert.assertEquals(textArea.getText(), actualText);
    }
}
\end{lstlisting}

\newpage \section{Системное тестирование}
\subsection{Общее описание}
Целью является тестирование всей системы в целом на реальной БД.

\newpage \subsection{Тестирование регистрации пользователя}
Протестируем работу GUI и работу с базой данных.  

\subsubsection{Попытка регистрации уже существующего пользователя}
Для начала проведём тестирования попытки регистрации пользователя, который уже есть в базе. Должно быть соответвующее сообщение. Класс для проведения этого тестирования представлен в листинге \ref{lstlisting_database:test_ui}.

\begin{lstlisting}[language=java, caption=код модуля DatabaseAndRegistryUiTest.java, label=lstlisting_database:test_ui]
package ui.RegistryUiTest;

import javafx.scene.control.Label;
import org.junit.Assert;
import org.junit.Test;
import ui.custom.LoginUiCustomTest;

import static org.loadui.testfx.GuiTest.find;

/**
 * Created by Алексей on 26.12.2016.
 */
public class DatabaseAndRegistryUiTest extends LoginUiCustomTest {

    @Test
    public void userExistsTest() {
        clickOn(registry);
        clickOn("#login").write("Aleksey");
        clickOn("#password").write("tesPass");
        clickOn("#signUp");
        Label outputErrorText = find("#outputErrorText");
        Assert.assertEquals(outputErrorText.getText(), "Пользователь с логином: Aleksey существует!");
        closeRegistryWindowAndTest();
    }


}
\end{lstlisting}

\subsubsection{Добавление пользователя в БД}
\par Сценарий тестирования регистрации пользователя заключается в следующем:
\begin{enumerate}
\item На форме авторизации нажатие кнопки "Регистрация".
\item Запоняем все поля формы, предварительно убедившись, что такого пользователя нет в базе.
\item Нажимаем на кнопку регистрация и убеждаемся, что в БД плявилась соответствующая запись. При этом с формы ргистрации мы должны обратно прейти к форме авторизации.
\item Вводим логин и пароль нового пользователя после чего должны попасть на главное окно.
\item Убеждаемся, что активированы только те методы, которые были указаны при регистации.
\end{enumerate}

\newpage \subsection{Тестирвоание загрузки и сохранения в файл}
Тестирование проводится вручную  по следущему сценарию:
\begin{enumerate}
\item Авторизуемся.
\item Пишем в поле ввода "Текст для записи в файл".
\item В меню выбираем Файл->Сохранить.
\item Убедимся, что файл был создан.
\item В меню выбираем Файл->Загрузить.
\item Убедимя, что файл загружен правильно.
\end{enumerate}

\newpage \subsection{Проведение тестирования}

В таблицах \ref{table:data_type1} - \ref{table:data_type5}, описано применение ручных сценариев тестирования.

\begin{table}[h]
	\caption{Тестрование регистрации пользователя}
	\centering
	\begin{tabular}{|c|c|c|}
	\hline 
	№  & Результат & Примечание \\ 
	\hline 
	1 & \includegraphics[scale=0.3]{img/database/before_registry_user.png} & В базе присутсвуют \\ && только два пользователя \\
	\hline 
	2 & \includegraphics[scale=0.3]{img/database/before_registry_methods.png} & В базе каждому пользователю \\ && соответсвуют два метода\\
	\hline 
	3 & \includegraphics[scale=0.3]{img/login_form.png} & Открываем окно авторизации\\
	\hline 
	4 & \includegraphics[scale=0.3]{img/database/before_registry_form.png} & Открываем окно регистрации \\ && и заполняем все поля\\
	\hline 
	5 & \includegraphics[scale=0.3]{img/database/after_registry_user.png} & После регистрации можно увидеть, \\ && что введенный пользователь появился в БД\\
	\hline
\end{tabular} 
\label{table:data_type1} 
\end{table}

\begin{table}[pt!]
	\caption{Тестрование регистрации пользователя}
	\centering
	\begin{tabular}{|c|c|c|}
	\hline 
	№  & Результат & Примечание \\ 
	\hline 
	6 & \includegraphics[scale=0.3]{img/database/after_registry_methods.png} & Так же можно увидеть, что \\ && для нового пользователя \\ && появились соответсвующие методы\\
	\hline 
	7 & \includegraphics[scale=0.3]{img/database/authorization.png} & Открываем окно авторизации \\ && и заходим в систему под \\ && новым пользователем\\
	\hline 
	8 & \includegraphics[scale=0.3]{img/database/main_open_methods.png} & Так же можно увидеть, \\ && что для нового пользователя появились\\ &&  соответсвующие методы шифрования\\
	\hline 
	9 & \includegraphics[scale=0.3]{img/database/main_open_methods2.png} & Так же можно увидеть, \\ && что для нового пользователя появились\\ &&  соответсвующие методы расшифрования\\
	\hline 
\end{tabular} 
\label{table:data_type2} 
\end{table}

\begin{table}[pt!]
	\caption{Тестрование сохранения в файл}
	\centering
	\begin{tabular}{|c|c|c|}
	\hline 
	№  & Результат & Примечание \\ 
	\hline 
	1 & \includegraphics[scale=0.3]{img/file/open/text_open4.png} & Вводим текст в поле для \\ && ввода текста (memo) \\
	\hline 
	2 & \includegraphics[scale=0.3]{img/file/save/text_path.png} & Выбираем путь, куда будет \\ && сохранен файл с текстом\\
	\hline 
	3 & \includegraphics[scale=0.3]{img/file/save/text_okey_save.png} & Вывод сообщения о том что файл сохранен\\
	\hline 
\end{tabular} 
\label{table:data_type3} 
\end{table}

\begin{table}[pt!]
	\caption{Тестрование открытия файла}
	\centering
	\begin{tabular}{|c|c|c|}
	\hline 
	№  & Результат & Примечание \\ 
	\hline 
	1 & \includegraphics[scale=0.3]{img/file/open/text_clear.png} & Выбираем очистку поля ввода \\
	\hline 
	2 & \includegraphics[scale=0.3]{img/file/open/text_clear1.png} & Видим что поле очистилось\\
	\hline 
	3 & \includegraphics[scale=0.3]{img/file/open/text_open.png} & Выбираем открытие файла\\
	\hline 
\end{tabular} 
\label{table:data_type4} 
\end{table}
\begin{table}[pt!]
	\caption{Тестрование открытия файла}
	\centering
	\begin{tabular}{|c|c|c|}
	\hline 
	№  & Результат & Примечание \\ 
	\hline 
	5 & \includegraphics[scale=0.3]{img/file/open/text_open2.png} & Выбираем путь до файла\\
	\hline 
	6 & \includegraphics[scale=0.3]{img/file/open/text_open3.png} & Видим, что файл открылся\\
	\hline
	7 & \includegraphics[scale=0.3]{img/file/open/text_open4.png} & Текст из файла в поле\\
	\hline
\end{tabular} 
\label{table:data_type5} 
\end{table}

\newpage\section*{Выводы}
\addcontentsline{toc}{section}{Выводы}
В ходе выполнения лабораторной работы было разработано и проведено комплексное тестирвание пректа. Для написания тестов была использована библиотека JUnit, а действия пользователя моделировались с помошью библиотеки TestFX. Ручные тесты составлялись из рассчёта на то, чтобы смоделировать типовые действие пользователя, которые не были протестированы на предыдущем этапе. В ходе тестирования ошибок не обнаружено.

\newpage\section*{Приложение 1. Исходный код тестируемой программы}
\addcontentsline{toc}{section}{Приложение 1. Исходный код тестируемой программы}

\newpage\section*{Приложение 2. Исходный код тестирующих классов}
\addcontentsline{toc}{section}{Приложение 2. Исходный код тестирующих классов}


\end{document}